\subsection{Komplexe Funktionen}
\subsubsection{Komplexe Zahl -> komplexe Analysis}

\paragraph{Definiton:} $z \in \mathbb{C}$: Tupel $(a,b)$ mit $a, b \in \mathbb{R}$ für die gilt:

Addition: $(a,b)\pm(u,v) = ((a \pm u),(b \pm v))$

Multplikation: $(a,b) (u,v) = ((au - bv),(av + bu))$

Körper:
\begin{itemize}
    \item Assoziativgesetzt und Kommutativgestz
    
    $\boxed{z_1 \times (z_2 \times z_3) = (z_1 \times z_2) \times z_3}$

    $\boxed{z_1 \times z_2 = z_2 \times z_1}$

    für für Addition ($\times \triangleq +$) und Multiplikation ($\times \triangleq  $).

    \item Distributivgesetz: $z_1 (z_2 + z_3) = z_1   z_2 + z_1   z_3$
\end{itemize}

Inverse: 
\begin{itemize}
    \item Addition: $z + (-z) = 0$
    \item Multiplikation: $z   \frac{1}{z} = 1$ $(z \neq 0)$
\end{itemize}

Neutrale: 
\begin{itemize}
    \item Addition: $(0,0)$
    \item Multiplikation: $(1,0)$
\end{itemize}

$\implies$ "unitärer Ring" (Aber: keine Anordnungseigenschaft.)

\paragraph{Definition:} $\boxed{z:= a + i b \text{ mit } i^2 = -1 \text{ } (a,b \in \mathbb{R})}$

\paragraph{Betrag} einer kompleen Zahl: uklidische Norm: $|z| := |(a,b)| = \sqrt{a^2+b^2}$

\paragraph{Definition:} Komplee Konjugation: $z^*:=a-ib$ zu $z = a + ib$

$\implies |z| = \sqrt{z   z^*}$

\underline{Realteil:} $Re(z) = \frac{1}{2}(z+z^*) = a$

\underline{Imaginärteil:} $Im(z) = \frac{1}{2i}(z-z^*) = b$

\paragraph{Potenzen:} 

$z^n = z   z   ...   z$ (n-mal)

$z^{-n} = \frac{1}{z}   \frac{1}{z}   ...   \frac{1}{z}$ (n-mal)

wobei: $\frac{1}{z} = \frac{z^*}{z   z^*} = \frac{x-iy}{x^2+y^2}$ für $z:=x+iy$ $x,y \in \mathbb{R}$

($\implies f(z) = \frac{1}{z} := u + iv$ Normalform,  $u,v \in \mathbb{R}$)

\newpage
\paragraph{Darstellung in 2D Ebene:} $z := x + i y$

\includegraphics[height=2.5cm]{mm2-01/01.png}

-> auch darstellbar durch Betrag $|z|$ und Winkel $\alpha$ ($:=$ "Argument").



\subsubsection{Komplexe Funktionen}
\begin{enumerate}
    \item \underline{Polynome} (vom Grad n): $P_n(z) = a_n z^n + a_{n-1} z^{n-1}+ ... + a_1 z + a_0$ mit $a_i \in \mathbb{C}, a_n \neq 0$; $n \in N_0$
    \underline{z.B.:} $w=f(z) = z^2 = (x+iy)^2 = (x^2-y^2) + i(2xy) = u + i v$
    \item \underline{Rationale Funktionen:} $\frac{P_n(z)}{Q_m(z)}$ mit Polynom $P_n, Q_m$
    \item \underline{Potenzreihen:} $P(z) = \sum\limits_{n=0}^\infty a_n z^n$ mit $a_n \in \mathbb{C}$

    -> Konvergenz ?

    relle Funkton: "Konvergenzradius"

    -> komplexe Funktionen: offene Kresscheibe um $0 in \mathbb{C}$ mit Radius $r$.
\end{enumerate}

\subsubitem{Geometrische Darstellung}

\begin{itemize}
    \item Darstellung von $u(x,y)$ und $v(x,y)$ separat als Funktionen von $x,y$ (Z.B "Höhenlinien" in 2D)
    \item Darstellung als Vektorfeld 
    $
    \begin{pmatrix}
        u(x,y)\\
        v(x,y)
    \end{pmatrix}$: 

    \includegraphics[height=2.5cm]{mm2-01/02.png}

    \item $(z,w)$ Darstellung:
    
    \includegraphics[height=2.5cm]{mm2-01/03.png}
\end{itemize}
\newpage
\underline{Beispiel:} $f(z) = z^2 = (x^2-y^2) + i(2xy)$

\includegraphics[height=2.5cm]{mm2-01/04.png}

\fbox{\parbox{0.9\linewidth}{Suche die Kurven in der z-Ebene, die auf ein orthogonals karteschies Netz in der w-Ebene abgebildet werden.}}

(hier: Scharen orthogonaler Hyprbahn.)
\includegraphics[height=2.5cm]{mm2-01/05.png}

\subsubsection{Exponentalfunktion}
\paragraph{Definition:} $\exp:\mathbb{C} \rightarrow \mathbb{C}, z \mapsto \sum\limits_{k=0}^\infty \frac{z^k}{k!}$

$f(z)=\exp(z)=e^z$ ist auf \underline{ganz $\mathbb{C}$} konvergent. 

(Beweis: analog wie in $\mathbb{R}$ mit Quotienten-Kriterium.)

\paragraph{Euler-Formel:} $\exp(i\alpha) = \cos(\alpha)+ i \sin(\alpha)$ $\forall \alpha \in \mathbb{R}$

\paragraph{Beweis:} 
\begin{align*}
    \exp(i\alpha) &= \sum\limits_{k=0}^\infty i^k\frac{\alpha^k}{k!} \\
    &= \sum\limits_{k=0}^\infty i^{2k}\frac{\alpha^{2k}}{(2k)!} + \sum\limits_{k=0}^\infty i^{2k+1}\frac{\alpha^{2k+1}}{(2k+1)!} \\
    &= \sum\limits_{k=0}^\infty (-1)^{k}\frac{\alpha^{2k}}{(2k)!} + i \sum\limits_{k=0}^\infty (-1)^{k}\frac{\alpha^{2k+1}}{(2k+1)!} \\
    &= \cos(\alpha) + i \sin(\alpha)
\end{align*}
Für die efinition von $\cos,\sin$.

Durch geeignete \underline{Definition} von $\sin(z)$ und $cos(z)$: $(\alpha \rightarrow z)$ Euler-Formel auch für $z\in \mathbb{C}$ gültig!

\subsubsection{Trigonometrisch Funktionen:}

$\cos(z) = \frac{1}{2}(e^{iz}+e^{-iz})$, $\sin(z) = \frac{1}{2i}(e^{iz} - e^{-iz})$ 

$\implies$
\begin{itemize}
    \item $\tan(z):=\frac{\sin(z)}{\cos(z)}=\frac{1}{\cot(z)}$
    \item $\sin^2(z) + \cos^2(z) = 1$
\end{itemize}

\paragraph{Hyperbelfunktionen:}

$\cosh(z) := \frac{1}{2}(e^z+e^{-z})$

$\sinh(z) := \frac{1}{2}(e^z-e^{-z})$

$\implies \cosh^2(z)+\sinh^2(z) = 1$

\paragraph{Exponentialfunktion und Polarsdarstellung:}

\includegraphics[height=2.5cm]{mm2-01/06.png}

$z = (a,b) = a + ib = \sqrt{a^2+b^2} (\cos(\alpha), \sin(\alpha)) = |z|\exp(i\alpha)$ wobei $\alpha := \arg(z)$ das "Argument" von $z$ ist.

\paragraph{Exponentialfunktion:} $e^w  e^z = w^{w  z}$ für $w,z \in \mathbb{C}$ auch: $e^{i\alpha}  e^{i\beta} = e^{i(\alpha + \beta)}$ für $\alpha, \beta \in \mathbb{R}$

Damit: $e^z = e^{x+iy} = e^x  e^{iy} = e^z  (\cos(y) + i \sin(y))$ mit $|e^z| = e^x$ und $ = \arg(e^z)$

(Beweis für $e^w\dot e^z = e^{w+z}$: siehe Lehrbücher)

\paragraph{Produkt:}$z   w=|z| |w|  e^{i(\alpha + \beta)}$ für $z:=|z|  e^{i\alpha}$ und $w:=|w|  e^{i\beta}$

-> Interpretation: Drehung um $\beta$ und Streckung um $|w|$ von $z$ .

(Achtung: in der Regel ist $z  w \in \mathbb{C} \implies$ \underline{kein} "Skalarprodukt"!)