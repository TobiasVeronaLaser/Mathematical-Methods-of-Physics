% Header: Here are all packages used and some additional definitions
%%%%%%%%%%%%%%%%%%%%%%%%%%%%%%%%%%%%%%%%%%%%%%%%%%%%%%%%%%%%%%%%%%%

\documentclass[11pt,a4paper]{scrartcl}
\usepackage[margin=2.5cm]{geometry}
\usepackage[onehalfspacing]{setspace}
\usepackage{graphicx} % zum Einbinden von Graphiken
\usepackage[breaklinks=true,colorlinks=true,linkcolor=blue,urlcolor=blue,citecolor=blue]{hyperref} % f. Referenzen
\usepackage{amsmath,amsthm,amssymb} % Mathematik Umgebung 
\usepackage{icomma} % Intelligentes Komma, das den richtigen Abstand zwischen Dezimalzahlen als auch in Formeln wählt.
\usepackage[ngerman]{babel} % Deutsche Bezeichnungen bei Inhaltsangabe etc
\usepackage[T1]{fontenc}    % andere Schriftsatzkodierung für richtige Silbentrennung bei Umlauten
\usepackage[locale = DE,space-before-unit=true,per-mode = symbol]{siunitx} % Bessere Einheiten
\usepackage{booktabs,multirow} % Pakete zur Erstellung von Tabellen
\usepackage{placeins} % Definiert den Befehl “\FloatBarrier”, der die Ausgabe der davor eingebundenen Bilder erzwingt, befor der Text weiter geht. (Mit vorsicht zu verwenden)
\usepackage[natbib,abbreviate=true,doi=false,style=numeric-comp,giveninits=true,sorting=none]{biblatex} % Modernes Paket zur Erzeugung von Bibliografien (benötigt biber!)
\usepackage{csquotes} % Fortgeschrittene Funktionen für Zitate, für die deutsche Form der Anführungszeichen bei Referenzen
%\addbibresource{MyBibliography.bib} % Ort der .bib Datei, die die Datenbank für Literatur/Referenzen enthält.

\graphicspath{{images/} {chapters/}}

\DeclareSIUnit{\dBm}{dBm}
\DeclareSIUnit[per-mode=reciprocal]\WN{\per\centi\meter}

%%%%%%%%%%%%%%%%%%%%%%%%%%%%%%%%%%%%%%%%%%%%%%%%%%%%%%%%%%%%%%%%%%%
\begin{document}
%
\titlehead{\includegraphics[width=5cm]{logo.jpg}}
\title{Mathematical Methods of Physics}
\author{Tobias Laser\thanks{\href{mailto:tobias.laser@uibk.ac.at}{tobias.laser@uibk.ac.at}}}
\date{\today}
\maketitle
\vfill
\renewcommand\abstractname{Kurzfassung}
\section*{\abstractname}
\textit{-||-}
\thispagestyle{empty}
%
%
\tableofcontents
\thispagestyle{empty}
\cleardoublepage
\pagenumbering{arabic} 
\newpage
%
%
\section{Funktionentheorie}
\newpage
\subsection{Komplexe Funktionen}
\subsubsection{Komplexe Zahl -> komplexe Analysis}

\paragraph{Definiton:} $z \in \mathbb{C}$: Tupel $(a,b)$ mit $a, b \in \mathbb{R}$ für die gilt:

Addition: $(a,b)\pm(u,v) = ((a \pm u),(b \pm v))$

Multplikation: $(a,b) (u,v) = ((au - bv),(av + bu))$

Körper:
\begin{itemize}
    \item Assoziativgesetzt und Kommutativgestz
    
    $\boxed{z_1 \times (z_2 \times z_3) = (z_1 \times z_2) \times z_3}$

    $\boxed{z_1 \times z_2 = z_2 \times z_1}$

    für für Addition ($\times \triangleq +$) und Multiplikation ($\times \triangleq  $).

    \item Distributivgesetz: $z_1 (z_2 + z_3) = z_1   z_2 + z_1   z_3$
\end{itemize}

Inverse: 
\begin{itemize}
    \item Addition: $z + (-z) = 0$
    \item Multiplikation: $z   \frac{1}{z} = 1$ $(z \neq 0)$
\end{itemize}

Neutrale: 
\begin{itemize}
    \item Addition: $(0,0)$
    \item Multiplikation: $(1,0)$
\end{itemize}

$\implies$ "unitärer Ring" (Aber: keine Anordnungseigenschaft.)

\paragraph{Definition:} $\boxed{z:= a + i b \text{ mit } i^2 = -1 \text{ } (a,b \in \mathbb{R})}$

\paragraph{Betrag} einer kompleen Zahl: uklidische Norm: $|z| := |(a,b)| = \sqrt{a^2+b^2}$

\paragraph{Definition:} Komplee Konjugation: $z^*:=a-ib$ zu $z = a + ib$

$\implies |z| = \sqrt{z   z^*}$

\underline{Realteil:} $Re(z) = \frac{1}{2}(z+z^*) = a$

\underline{Imaginärteil:} $Im(z) = \frac{1}{2i}(z-z^*) = b$

\paragraph{Potenzen:} 

$z^n = z   z   ...   z$ (n-mal)

$z^{-n} = \frac{1}{z}   \frac{1}{z}   ...   \frac{1}{z}$ (n-mal)

wobei: $\frac{1}{z} = \frac{z^*}{z   z^*} = \frac{x-iy}{x^2+y^2}$ für $z:=x+iy$ $x,y \in \mathbb{R}$

($\implies f(z) = \frac{1}{z} := u + iv$ Normalform,  $u,v \in \mathbb{R}$)

\newpage
\paragraph{Darstellung in 2D Ebene:} $z := x + i y$

\includegraphics[height=2.5cm]{mm2-01/01.png}

-> auch darstellbar durch Betrag $|z|$ und Winkel $\alpha$ ($:=$ "Argument").



\subsubsection{Komplexe Funktionen}
\begin{enumerate}
    \item \underline{Polynome} (vom Grad n): $P_n(z) = a_n z^n + a_{n-1} z^{n-1}+ ... + a_1 z + a_0$ mit $a_i \in \mathbb{C}, a_n \neq 0$; $n \in N_0$
    \underline{z.B.:} $w=f(z) = z^2 = (x+iy)^2 = (x^2-y^2) + i(2xy) = u + i v$
    \item \underline{Rationale Funktionen:} $\frac{P_n(z)}{Q_m(z)}$ mit Polynom $P_n, Q_m$
    \item \underline{Potenzreihen:} $P(z) = \sum\limits_{n=0}^\infty a_n z^n$ mit $a_n \in \mathbb{C}$

    -> Konvergenz ?

    relle Funkton: "Konvergenzradius"

    -> komplexe Funktionen: offene Kresscheibe um $0 in \mathbb{C}$ mit Radius $r$.
\end{enumerate}

\subsubitem{Geometrische Darstellung}

\begin{itemize}
    \item Darstellung von $u(x,y)$ und $v(x,y)$ separat als Funktionen von $x,y$ (Z.B "Höhenlinien" in 2D)
    \item Darstellung als Vektorfeld 
    $
    \begin{pmatrix}
        u(x,y)\\
        v(x,y)
    \end{pmatrix}$: 

    \includegraphics[height=2.5cm]{mm2-01/02.png}

    \item $(z,w)$ Darstellung:
    
    \includegraphics[height=2.5cm]{mm2-01/03.png}
\end{itemize}
\newpage
\underline{Beispiel:} $f(z) = z^2 = (x^2-y^2) + i(2xy)$

\includegraphics[height=2.5cm]{mm2-01/04.png}

\fbox{\parbox{0.9\linewidth}{Suche die Kurven in der z-Ebene, die auf ein orthogonals karteschies Netz in der w-Ebene abgebildet werden.}}

(hier: Scharen orthogonaler Hyprbahn.)
\includegraphics[height=2.5cm]{mm2-01/05.png}

\subsubsection{Exponentalfunktion}
\paragraph{Definition:} $\exp:\mathbb{C} \rightarrow \mathbb{C}, z \mapsto \sum\limits_{k=0}^\infty \frac{z^k}{k!}$

$f(z)=\exp(z)=e^z$ ist auf \underline{ganz $\mathbb{C}$} konvergent. 

(Beweis: analog wie in $\mathbb{R}$ mit Quotienten-Kriterium.)

\paragraph{Euler-Formel:} $\exp(i\alpha) = \cos(\alpha)+ i \sin(\alpha)$ $\forall \alpha \in \mathbb{R}$

\paragraph{Beweis:} 
\begin{align*}
    \exp(i\alpha) &= \sum\limits_{k=0}^\infty i^k\frac{\alpha^k}{k!} \\
    &= \sum\limits_{k=0}^\infty i^{2k}\frac{\alpha^{2k}}{(2k)!} + \sum\limits_{k=0}^\infty i^{2k+1}\frac{\alpha^{2k+1}}{(2k+1)!} \\
    &= \sum\limits_{k=0}^\infty (-1)^{k}\frac{\alpha^{2k}}{(2k)!} + i \sum\limits_{k=0}^\infty (-1)^{k}\frac{\alpha^{2k+1}}{(2k+1)!} \\
    &= \cos(\alpha) + i \sin(\alpha)
\end{align*}
Für die efinition von $\cos,\sin$.

Durch geeignete \underline{Definition} von $\sin(z)$ und $cos(z)$: $(\alpha \rightarrow z)$ Euler-Formel auch für $z\in \mathbb{C}$ gültig!

\subsubsection{Trigonometrisch Funktionen:}

$\cos(z) = \frac{1}{2}(e^{iz}+e^{-iz})$, $\sin(z) = \frac{1}{2i}(e^{iz} - e^{-iz})$ 

$\implies$
\begin{itemize}
    \item $\tan(z):=\frac{\sin(z)}{\cos(z)}=\frac{1}{\cot(z)}$
    \item $\sin^2(z) + \cos^2(z) = 1$
\end{itemize}

\paragraph{Hyperbelfunktionen:}

$\cosh(z) := \frac{1}{2}(e^z+e^{-z})$

$\sinh(z) := \frac{1}{2}(e^z-e^{-z})$

$\implies \cosh^2(z)+\sinh^2(z) = 1$

\paragraph{Exponentialfunktion und Polarsdarstellung:}

\includegraphics[height=2.5cm]{mm2-01/06.png}

$z = (a,b) = a + ib = \sqrt{a^2+b^2} (\cos(\alpha), \sin(\alpha)) = |z|\exp(i\alpha)$ wobei $\alpha := \arg(z)$ das "Argument" von $z$ ist.

\paragraph{Exponentialfunktion:} $e^w  e^z = w^{w  z}$ für $w,z \in \mathbb{C}$ auch: $e^{i\alpha}  e^{i\beta} = e^{i(\alpha + \beta)}$ für $\alpha, \beta \in \mathbb{R}$

Damit: $e^z = e^{x+iy} = e^x  e^{iy} = e^z  (\cos(y) + i \sin(y))$ mit $|e^z| = e^x$ und $ = \arg(e^z)$

(Beweis für $e^w\dot e^z = e^{w+z}$: siehe Lehrbücher)

\paragraph{Produkt:}$z   w=|z| |w|  e^{i(\alpha + \beta)}$ für $z:=|z|  e^{i\alpha}$ und $w:=|w|  e^{i\beta}$

-> Interpretation: Drehung um $\beta$ und Streckung um $|w|$ von $z$ .

(Achtung: in der Regel ist $z  w \in \mathbb{C} \implies$ \underline{kein} "Skalarprodukt"!)
\newpage
\subsection{Komplexe Umkehrfunktionen}
\subsubsection{Eindeutige Funktionen}

Eine komplexe Funktion heißt \underline{eindeutig}, wenn es für jeden Wert $z \in D \subset \mathbb{C}$ genau einen Funktionswert $w = f(z) \in \mathbb{C}$ gibt. ($D:=$ Definitionsbereich).

Beispiel: $f(z) = z^2$, $f(z)=e^z$

\underline{Mehrwertige} Funktionen: z.B. $f(z)=\sqrt{z}$, $f(z)=\log(z)$

(-> Verzweigungsspunkte/ Schnitte)

\paragraph{Eine} komplexe Funktion ist \underline{injektiv}, wenn für $z_1 \neq z_2$ folgt, dass auch $f(z_1) \neq f(z_2)$. (Bzw. $f(z_1) = f(z_2) \implies z_1 = z_2$).

\underline{Beispiel:} $f(z) = z^2$ ist \underline{eindeutig}, aber nicht \underline{injektiv}.

\underline{surjektiv:} für $f:z\in D\subset\mathbb{C} \rightarrow w\in W \subset \mathbb{C}$ gitl: fpr jedes $w \in W$ existiert ein $z \in D$, sodass $f(z) = w$. -> dann existiert eine umkehrfunktion $f^{-1}: W \rightarrow D$, die ebenfalls eindeutig ("wohldefniert") ist.

(\underline{Bijektiv: sowohl injektiv als auch surjektiv}).

\underline{Beispiel}
\begin{itemize}
    \item $f(z) = a z + b$, $a \neq 0$: injektiv und surjektiv
    \item $f(z) = z^2$, nicht injektiv(da $f(1) = f(-1) = 1$), aber surjektiv (da in $\mathbb{C} jede Wurzel lösbar ist$).
\end{itemize}

Die komplexe Exponentialfunktion ist \underline{nicht injektiv}:

$f(z) = e^z = w$ mit$w := |w| (\cos(\alpha) + i \sin(\beta))$ mit $\alpha := \psi + 2\pi k$, $k \in \{0, \pm1, \pm2, ...\}$ $\implies$ 
\begin{align*}
    w =& |w|  [(\cos(\psi)\dot \cos(2\pi k) - \sin(\psi) \sin(i2 \pi k)+i(\sin(\psi)\cos(s\pi k) + \cos(\psi) \sin(2 \pi k)))] \\
    =& |w|(\cos(\psi)+ i \sin(\psi))
\end{align*}

\underline{Abbldung:} $z \rightarrow w = e^z = e^x (\cos(y)+i\sin(y))$

Exemplarisch: eine vertikale Gerade in $D \subset \mathbb{C}$ mit $Re(z) = x :=$ konstant f+r ein $x\in\mathbb{R}$

\includegraphics[height=2.5cm]{mm2-02/01.png}

$\rightarrow$ Abbildung auf $W$?

$\rightarrow$ Kreis um $0$ mit Radius $e^x$:

\includegraphics[height=2.5cm]{mm2-02/02.png}

$\implies$ In $y\in\mathbb{R}$ ist diese Abbildung $2\pi$-periodisch

\underline{Definition:} "geschlitzte Ebene": $\mathbb{C}^-$

\underline{hier z.B.:} $\mathbb{C}^-:=\mathbb{C}\setminus\{w \in \mathbb{C}: \Im(W) = 0, \Re(W)\leq 0\} \Leftrightarrow$ komplexe Exponentialfunktion ist \underline{bijektiv} in $\{z\in\mathbb{C} : -\pi < \Im(z) < \pi\}$

darin: Umkehrfunktion definierbar: "Hauptzweig der Logarithmusfunktion"

\subsubsection{Komplexer Hauptzweig - Logarithmus}

Für $z\in\mathbb{C}^-:$ (\underline{hier:} $-\pi < y < +\pi, x \in \mathbb{R}$) gilt:

$z =  x + i y = \ln(e^z)$ $\leftrightarrow \ln$ Umkehrfunktion von $\exp$

$= \ln(e^x  e^{iy})$

Sei $f(z)=\ln(z) = w = u + i v$

$\leftrightarrow z = e^w = e^{u + i v} = e^u  e^{iv} = |z| e^{i \alpha}$

mit $|e^w| = e^u = |z| \implies u = \ln(z)$

und $\arg(z) = \arg(e^w) = v =: \alpha$

$\implies w = u + iv = ln|z| + i \arg(z) = \ln(z)$

$\forall  x \in  \mathbb{R}$ un $y \in (-\pi, \pi)$ ! (eindeutig $\leftrightarrow$ "Hauptzweig")

$\implies$ damit gilt auch: $\ln(z^*) = (\ln(z))^*$ $\forall z \in \mathbb{C}^-$

\underline{Vorsicht:} im Allgemeinen ist $\ln(z  w) \neq \ln(z)   \ln(w)$

\underline{Beispiel:} $z := 1 + i \implies |z| = \sqrt{2} \implies z = \sqrt{2}(\frac{1}{\sqrt{2}} + \frac{1}{\sqrt{2}})$

$\implies z = \sqrt{2}(\cos\frac{\pi}{4} + i \sin\frac{\pi}{4}) = \sqrt{2}\exp(i\frac{\pi}{4})$

$\implies z = \ln(1 + i) = (\ln\sqrt{2}) + i (\frac{\pi}{4})$

\underline{Beispiel:} Für $\ln(z   w) \neq \ln(z) + \ln(w)$:

\begin{itemize}
    \item $z := 1 + i \implies \ln(z) = \ln(\sqrt) + i(\frac{\pi}{4})$ (siehe vorhin)
    \item $w := -1$
\end{itemize}

\underline{Hier:} geschlitzte Ebene mit  $\psi\in(-\pi, \pi]$ für $w:=|w|  e^{i\psi}$

\includegraphics[height=2.5cm]{mm2-02/03.png}

$\implies \ln(w) = \ln(1) + i\pi = 0 + i\pi$

$\implies \ln(w) + \ln(z)  = \ln(\sqrt{z}) + i\frac{5\pi}{4}$

\underline{Aber:} $\ln(w   z) = \ln(-1 -i) = \ln(\sqrt{z})  + i\frac{3\pi}{4}$

\includegraphics[height=2.5cm]{mm2-02/04.png}

\underline{Weiters:} (einfachstes) Beispiel: $w=-1$, $z=-1$

\includegraphics[height=2.5cm]{mm2-02/05.png}

$\ln(w) + \ln(z) = 2 \dot(0 + i\pi) = 2\pi  i$

$\ln(w   z) = \ln(+1) = 0$

(Jeweils wenn nach üblicher Konvention er "Schlitz" der Logarithmusfunktion auf der negativen reellen Achse definiert ist und $-\pi < \alpha \leq \pi$ gilt.)

\subsubsection{Komplexe Wurzeln und ihre Hauptzweig}

Sei $w \in \mathbb{C}$ mit $w = |w|\exp(i\alpha)$ mit $\alpha \in [0, 2pi)$.

$rightarrow$ Für welche $z\in\mathbb{C}$ gilt: $t^n = w$ ? $\Leftrightarrow$ "$z:=\sqrt[n]{w}$"?

$\implies z_k = \sqrt[n]{|w|}   \exp(i\frac{\alpha+2 \pi k}{n})$ mit $k \in \{0, 1, ... n - 1\}$

d.h. obwohl $w = |w|(\cos\alpha + i \sin\alpha):=|w|(\cos(\alpha+2\pi k) + i \sin(\alpha+2\pi k))$ ist $\sqrt[n]{w}$ \underline{nicht} eindeuting, sondern mehrwertig $\rightarrow$ $z_k$.

Für $w := 1$: n-te "Einheitwurzeln": $E_n = \exp(2\pi o \frac{k}{n})$ mit $k \in {0,1, ..., n-1}$

$\implies z_k=\sqrt[n]{|w|}\exp(i\frac{\alpha}{n})E_n(k)$

\underline{Allgemeinen}: $z\rightarrow z^n$ für $n\in\mathbb{n}, n\geq 2$ auf $\mathbb{C}$ \underline{nicht} injektiv.

\underline{Einschränkung} auf Sektor $\{z \in \mathbb{C}| z = |z|e^{i\alpha} \text{ mit } -\frac{\pi}{n} < \alpha < \frac{\pi}{n}\}$ \underline{injektiv}

$\implies$ \underline{hierin} Umkehrfunktion $:=$ "\underline{Hauptzweig} der n-ten Wurzelfunktion $:= \sqrt[n]{w}$

Mehrdeutigkeit der Wurzelfunktion: Darstellung durch "Riemann'sche Blätter".

\underline{Beispiel:} $w=f(z):=\sqrt{z} = \sqrt{|z|}e^{i\frac{\alpha + 2 \pi k}{2}}$ mit $k\in{0, 1}$ und $w := |w| e ^ {i \psi}$

$\rightarrow$ \underline{Hauptzweig:} $k:=0 \Leftrightarrow$ Umkehrfunktion eindeutig.

\includegraphics[height=2.5cm]{mm2-02/06.png}

Für $\alpha = 0 + \epsilon \implies \psi = 0 + \epsilon \stackrel{\epsilon \rightarrow 0}{\implies} \cos\psi = +1$, $\sin\psi = 0$

aber $\alpha = 2\pi - \epsilon \implies \psi = \pi - \epsilon \stackrel{\epsilon \rightarrow 0}{\implies} \cos\psi = -1$, $\sin\psi = 0$

$\implies \lim\limits_{y\downarrow 0} \sqrt{x+i y} = \sqrt{x} = -\lim\limits_{y\uparrow 0} \sqrt{x + i y}$

\includegraphics[height=2.5cm]{mm2-02/07.png}

\underline{Mehrdeutigkeit des kompleen Logarithmus:}

$\ln z = \ln |z| + i \arg z + i 2 \pi n =: \ln |z| + i \alpha_0 + i 2 \pi n$, $n \in \mathbb{Z}$

\includegraphics[height=2.5cm]{mm2-02/08.png}

\underline{Problem} (Siehe später...): geschlossene Kurven bei kompleen kurvenintralen ?!

\subsubsection{Komplexe Potenzfunktion}

Für $a\in \mathbb{C}$, $z\in \mathbb{C}^-$:

$P_a: z \rightarrow r^\alpha := \exp(a \ln z) \Leftrightarrow$ \underline{Hauptzweig}

\underline{Beispiel:} $i^i = \exp(i\ln i) = \exp(i\ln |i| + i i \arg i) = \exp(0+ i^2 \frac{\pi}{2}) = \exp(-\frac{pi}{2})$

Es gilt: $z^a z^b = z^{a+b}$ $\forall a,b \in \mathbb{C}$, $z \in \mathbb{C}^-$ 
\newpage
\input{chapters/mm2-03-complex_derivation.tex}
\newpage
\input{chapters/mm2-04-complex_integration.tex}
\newpage
\newpage
\input{chapters/mm2-05-complex_power_series.tex}
\newpage
\input{chapters/mm2-06-residue_theorem.tex}
\newpage
\input{chapters/mm2-07-improper_real_integration.tex}
\newpage
\subsection{Integrale um Schnitte}
\subsubsection{Endliche Schnitte}
\underline{Beispiel:} $f(z) = \ln(\frac{z + 1}{z - 1})$ für $z \neq \pm 1$

Wähle Schnitt auf reeler Achse ($y=0$) mit $-1 < x < +1$: darin ist $\frac{z+1}{z-1}$ reell und negativ, und $-\pi < \arg(\frac{z+1}{z-1}) \leq + \pi$.
\paragraph{Aufgabe:} bestimme $\oint\limits_{C_r(0)} f(z) dz$ für Kreis um $0$ mit $R>1$.

\includegraphics[height=2.5cm]{chapters/mm2-08/mm2-08-01.png}

Sei $z+1:=r_1e^{i\theta_1}$, $z+1:=r_2e^{i\theta_2}$ bzw. $z =-1 + r_1e^{i\theta_1}$, $z =+1 + r_2e^{i\theta_2}$

\includegraphics[height=2.5cm]{chapters/mm2-08/mm2-08-02.png}

$\implies \ln\frac{z+1}{z-1} = \ln|\frac{r_1}{r_2}| + i(\theta_1-\theta_2+2\pi m)$ 

hier: Hauptwert -> $m:=0$

$\theta_1, \theta_2 \in (-\pi, \pi]$

Zuerst: geschlossene Kurve außerhalb des Schnitts:

\includegraphics[height=2.5cm]{chapters/mm2-08/mm2-08-03.png}

$\implies \oint\limits_c f(z) dz = 0$


$\implies \oint\limits_{C_{R>1}(0)} f(z) dz = -\oint\limits_{C_2} f(z) dz =
-\oint\limits_{C_\epsilon(-1)} f(z) dz -\oint\limits_{C_\epsilon(+1)} f(z) dz +
\lim\limits_{\epsilon \rightarrow 0} \int\limits_{+1}^{-1} f(x - i \epsilon) dx +
\lim\limits_{\epsilon \rightarrow 0} \int\limits_{-1}^{+1} f(x + i \epsilon) dx
(+\int\limits_{+1\Leftrightarrow r} f(z) dz)$

1: $C_\epsilon(+1):$ $f(z)?\frac{z + 1}{z - 1} \stackrel{z \rightarrow +1}{\approx} \ln(2) - \ln(z - 1)$ ($z\neq 1$)

$\oint\limits_{c_\epsilon(1)} \ln(z-1) dz = [z := 1 + \epsilon e^{it}] = \int\limits_0^{2\pi} \ln(\epsilon e^{it}) \epsilon i e^{it} dt \approx \int\limits_0^{2\pi} O(\epsilon\ln\epsilon) dt \rightarrow 0$

wobei $\ln(\epsilon e^{it}) = \ln|\epsilon e^{it}| + it + \ln(\epsilon) + it$

$\lim\limits_{\epsilon \rightarrow 0} \epsilon \ln(\epsilon) = \lim\limits_{\epsilon \rightarrow 0} \frac{\ln(\epsilon)}{\frac{1}{\epsilon}} \stackrel{\text{l'Hospital}}{=} \lim\limits_{\epsilon \rightarrow 0} \frac{\frac{1}{\epsilon}}{-\frac{1}{\epsilon^2}} = \lim\limits_{\epsilon \rightarrow 0} - \epsilon \rightarrow 0$

2. $C_{\epsilon}:\int\limits_{C_\epsilon(-1)}f(z)d(z)\approx\oint\limits_{C_\epsilon(-1)}\ln(z+1)-\ln(-z)$ ($z\neq-1$)

$\oint[C_\epsilon^*(+1)+C_\epsilon(-1)]=0$

tragen nichts bei.

$\implies \oint\limits_{C_{R>1}(0)} f(z)dz$ hängt nur von Differenz der Funktionswerte entlang des Schnitts ab.

$\implies \oint\limits_{C_{R>1}(0)} f(z)dz = ... = \lim\limits_{\epsilon\rightarrow 0}\int\limits_{-1}^{+1} [f(x+i\epsilon)] - f(x-i\epsilon) dx$

\includegraphics[height=2.5cm]{mm2-08/mm2-08-04.png}

$y:=0\pm\epsilon$, $x \in (-1, +1)$

$\implies r_1 = 1+x$, $R_2 = 1 - x$

Argument: $- \pi < \theta_1, \theta_2 \leq + \pi$.

I) $x < -1$: $+\epsilon$: $\theta_1 = \pi$, $\theta_2 = \pi \implies \Im(f) = \theta_1 - \theta_2 = 0$ (für $m:=0$)

\includegraphics[height=2.5cm]{mm2-08/mm2-08-05.png}

$-\epsilon$: $\theta_1 = -\pi$, $\theta_2 = -\pi \implies \Im(f) = 0$

\includegraphics[height=2.5cm]{mm2-08/mm2-08-06.png}

II) $-1 < x < 1$: $+\epsilon$: $\theta_1 = 0$, $\theta_2 = \pi \implies \Im(f) = -\pi$

\includegraphics[height=2.5cm]{mm2-08/mm2-08-07.png}

$-\epsilon$: $\theta_1 = 0$, $\theta_2 = -\pi \implies \Im(f)=+\pi \neq -\pi$

\includegraphics[height=2.5cm]{mm2-08/mm2-08-08.png}

III) $x > +1$: $\pm\epsilon \implies \Im(f) = 0$ (analog zu I))


\includegraphics[height=2.5cm]{mm2-08/mm2-08-09.png}

$\implies f(x + i\epsilon) = \ln|\frac{r_1}{r_2}| + i (-pi)$ für $-1 < x < +1$

$f(x - i\epsilon) = \ln|\frac{r_1}{r_2}| + i pi$

mit $r_1=|x-(-1)| = |x + 1| > 0$

$r_2=|x-(+1)| = |x - 1| > 0$

$\implies f(x + i\epsilon) - f(x - i\epsilon) = - 2 \pi i$

$\implies \oint\limits_{C_{R>1}(0)} \ln(\frac{z+1}{z-1}) dz = -\int\limits_{-1}^{+1} [f(x + i\epsilon) - f(x - i\epsilon)] dx = 4\pi i$. mit 

\includegraphics[height=2.5cm]{mm2-08/mm2-08-10.png}

\subsubsection{Unendliche Schnitte}

\paragraph{Beispiel} $\int\limits_{0}^{\infty} \frac{\sqrt{x}}{1+^2} dx = ? \Leftrightarrow \oint\limits_{C_{R\rightarrow \infty}} \frac{\sqrt{z}}{1+z^2} dz$, wobei $\frac{\sqrt{x}}{1+x^2} \stackrel{R\rightarrow\infty}{\rightarrow} \frac{x^{\frac{1}{2}}}{x^2} = x^{-\frac{3}{2}}$
 störker als $\frac{1}{}$ abfällt.

\includegraphics[height=2.5cm]{mm2-08/mm2-08-11.png}

$C_\epsilon(0)$: $t(z):=\epsilon e^{it}$ mit $t\in[0, 2\pi)$:

$\oint\limits_{C_\epsilon} \frac{\sqrt{z}}{1+z^2} dz = \int\limits_{0}^{2\pi} \frac{\sqrt{\epsilon e^{it}}}{1+(\epsilon e^{it})} i\epsilon e^{it} dt = i e^{\frac{3}{2}}\int\limits_{0}^{2\pi} \frac{e^{r i \frac{1}{2} t}}{1 + \epsilon^2 e^{2i t}} dt$

$\rightarrow 0$ für $\epsilon \rightarrow 0$

$\int\limits_{0}^{\infty} \frac{\sqrt{x}}{1+x} dx \widehat{=} \int\limits_{0}^{\infty} \frac{\sqrt{z}}{1 + z^2} dz$

wobei $\int\limits_{\infty}^{0} \frac{\sqrt{z}}{1 + z^2} dz = \int\limits_{\infty}^{0} \frac{\sqrt{r e^{2 \pi i}}}{1 + r^2 e^{4 \pi i}} dr = $ [für $z := r e^{i\phi}$ mit $\phi := 2 \pi$, da $e^{i\pi} = \cos(\pi) + i \sin(\pi) = -1$] $=\int\limits_{0}^{\infty} \frac{\sqrt{r}}{1 + r^2} dr = \int\limits_{0}^{\infty} \frac{\sqrt{x}}{1+x^2} dx \implies$ identisch.

$\implies \oint\limits_{C} \frac{\sqrt{z}}{1+z^2} dz = 2 \cdot \int\limits_{0}^{\infty} \frac{\sqrt{x}}{1+^2} dx + 0$

($\neq 0$ da zwei einfache Pole enthalten!)

$\operatorname{Res}[\frac{\sqrt{z}}{1+z^2}]{z = +i} = \frac{\sqrt{i}}{2i} = \frac{\sqrt{2}}{4i}(1+i)$

$\operatorname{Res}[\frac{\sqrt{z}}{1+z^2}]_{z = -i} = \frac{\sqrt{-i}}{2i} = \frac{\sqrt{2}}{4i}(1-i)$

wobei Hauptwert: $\sqrt{\pm i} = \sqrt{|i|} e^{\pm i \frac{\pi}{2}} = \cos(\pm\frac{\pi}{2}) + i \sin(\pm\frac{\pi}{2}) = \frac{1}{\sqrt{2}}(1 \pm i)$

und $\operatorname{Res}[\frac{\sqrt{z}}{(z + i)(z - i)}]_{z = \pm i} = \lim\limits_{z\rightarrow \pm i} (z \pm i)\frac{\sqrt{z}}{(z + 1)(z - i)} = \frac{\sqrt{\pm i}}{2 i}$

$\implies \int\limits_{0}^{\infty} \frac{\sqrt{x}}{1 + x^2} dx = \frac{1}{2} 2 \pi i [\frac{\sqrt{2}}{4i}(1 + i) + \frac{\sqrt{2}}{4 i}(1 - i)] = \frac{1}{2}\sqrt{2}\pi$
\newpage
\input{chapters/mm2-09-partial_differential_equation.tex}
\newpage
\input{chapters/mm2-10-laplace-equation.tex}
\newpage
\end{document}
